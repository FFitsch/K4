\documentclass[titlepage]{article}
\usepackage{fancyhdr}
\usepackage[margin=1.2in]{geometry}

\usepackage[utf8]{inputenc}
\usepackage[english]{babel}
\usepackage{csquotes}

\usepackage[backend=biber,style=apa,sorting=ynt]{biblatex}
\addbibresource{W2.bib}

\usepackage[breaklinks]{hyperref}

\usepackage{graphicx}
\usepackage{float}

\usepackage{derivative}

\usepackage{array,tabularx,calc}
\newlength{\conditionwd}
\newenvironment{conditions}[1][where:]{
        #1\tabularx{\linewidth-\widthof{#1}}[t]
        {>{$}l<{$} @{${}={}$} X@{}}}
  {\endtabularx\\[\belowdisplayskip]}


% Title Page -----------------------------------------------------------
\title{Protocol \\ K4 - Oxalic Acid Oxidation}
\author{Group F\\Jonas Adamer (12225913)\\Florian Fitsch (12218283)\\Leonhard Ritt (12208881)}
\date{Date of experiment: 2024/10/21\\Date of submission:}

\pagestyle{fancy}
\fancyhf{}
\fancyhead[R]{\thepage}
\fancyfoot[L]{K4 Oxalic Acid Oxidation}
\fancyfoot[C]{Group F}
\fancyfoot[R]{Adamer, Fitsch, Ritt}

\begin{document}

% TITLE PAGE
\maketitle
\thispagestyle{empty}

% TABLE OF CONTENTS
\newpage
\tableofcontents
\thispagestyle{fancy}


\newpage
\section{Objective}
In this assignment, the kinetics of chemical reactions are studied through two separate experiments.

In the first experiment, the redox reaction between oxalic acid and potassium permanganate is studied: After determining the absorption maximum of potassium permanganate, a series of solutions is produced and measured as a calibration for concentration of permanganate as a function of absorption. Finally, the change in permanganate-concentration is measured in-situ, both with and without a catalyst, in order to determine the reaction order.

In the second experiment, the acid-base reaction between phenolphthalein and sodium hydroxide is studied. The absorption of the reaction mixture is also measured in-situ and the reaction order regarding phenolphthalein determined.

\section{Experiment}
\subsection{Determination of Absorption Maximum of KMnO\texorpdfstring{\textsubscript{4}}{4}}
In order to find the absorption maximum, 79~mg of potassium permanganate were weighed out, dissolved in deionized wate and diluted in a 50~mL volumetric flask, yielding a 0.01~M solution. 5~mL of this solution were taken out and once again diluted in a 50~mL flask, yielding a concentration of 0.001~M.

2~mL of the prepared solution were filled into a cuvette and inserted into the photometer. Its Absorption was then measured in wavelength increments of 10~nm between 450~nm and 600~nm. For each wavelength, the photometer was calibrated by setting values of 0\% and 100\% transmission, using an opaque block and a cuvette filled with deionized water, respectively. Since high absorptions were measured at 530~nm and 550~nm, another series of measurements was taken between 520~nm and 560~nm with increments of 2~nm. Using this process, a maximum at 532~nm was determined.

\subsection{Creation of Calibration Curve for KMnO\texorpdfstring{\textsubscript{4}}{4} Concentration}



% \printbibliography

\end{document}
